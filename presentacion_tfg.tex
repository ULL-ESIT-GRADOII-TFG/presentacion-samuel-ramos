\documentclass{beamer}
%\documentclass[xcolor=dvipsnames]{beamer}
\usepackage[spanish]{babel}
\usepackage[utf8]{inputenc}
\usepackage{graphicx}
\usepackage{latexsym}

\newcommand{\beamer}{\textsc{beamer}}
\newtheorem{definicion}{Definición}
\newtheorem{ejemplo}{Ejemplo}

%%%%%%%%%%%%%%%%%%%%%%%%%%%%%%%%%%%%%%%%%%%%%%%%%%%%%%%%%%%%%%%%%%%%%%%%%%%%%%%
\title[Trabajo de Fin de Grado]{CodeLab\\
A Tool to automate repository creation and access control, giving support to distribute starter code, collect assignments and evaluate the students work on GitHub.}

\author[Samuel Ramos Barroso] {
Autor: Samuel Ramos Barroso \\
Director: Casiano Rodríguez León
}

\institute[ULL]{Escuela Superior de Ingeniería y Tecnología \\
                Departamento de Ingeniería Informática y de Sistemas \\
                Universidad de La Laguna}
\date[14-06-2018]{14 de Junio de 2018}
%%%%%%%%%%%%%%%%%%%%%%%%%%%%%%%%%%%%%%%%%%%%%%%%%%%%%%%%%%%%%%%%%%%%%%%%%%%%%%%

%\usetheme{Berlin}
\usetheme{Madrid}

%%%%%%%%%%%%%%%%%%%%%%%%%%%%%%%%%%%%%%%%%%%%%%%%%%%%%%%%%%%%%%%%%%%%%%%%%%%%%%%
\definecolor{pantone254}{RGB}{92,6,140}
\definecolor{pantone3015}{RGB}{92,6,140}
\definecolor{pantone432}{RGB}{92,6,140}
\setbeamercolor*{palette primary}{use=structure,fg=white,bg=pantone254}
\setbeamercolor*{palette secondary}{use=structure,fg=white,bg=pantone3015}
\setbeamercolor*{palette tertiary}{use=structure,fg=white,bg=pantone432}
\setbeamercolor*{palette sidebar primary}{use=structure,fg=pantone254}
\setbeamercolor*{palette sidebar tertiary}{use=structure,fg=pantone3015}
\setbeamercolor*{block title}{bg=pantone3015,fg=white}
\setbeamercolor*{alerted text}{fg=pantone432}
\setbeamercolor*{item projected}{fg=pantone254}
\setbeamercolor*{section in toc shaded}{use=structure,fg=structure.fg}
\setbeamercolor*{section in toc}{fg=pantone3015}
\setbeamercolor*{subsection in toc shaded}{fg=pantone3015}
\setbeamercolor*{subsection in toc}{fg=pantone432}

%%%%%%%%%%%%%%%%%%%%%%%%%%%%%%%%%%%%%%%%%%%%%%%%%%%%%%%%%%%%%%%%%%%%%%%%%%%%%%%
\begin{document}
  
%++++++++++++++++++++++++++++++++++++++++++++++++++++++++++++++++++++++++++++++  
\begin{frame}

  \includegraphics[width=0.3\textwidth]{img/ull.eps}
  \hspace*{7.5cm}
  \titlepage

\end{frame}
%++++++++++++++++++++++++++++++++++++++++++++++++++++++++++++++++++++++++++++++  

%++++++++++++++++++++++++++++++++++++++++++++++++++++++++++++++++++++++++++++++  
\begin{frame}
  \frametitle{Índice}  
  \tableofcontents
\end{frame}
%++++++++++++++++++++++++++++++++++++++++++++++++++++++++++++++++++++++++++++++  

\section{Introducción}
\begin{frame}[allowframebreaks,fragile]
  \frametitle{Introducción}
  
  \begin{center}
    CodeLab es una plataforma web basada en Javascript destinada al apoyo del profesorado para la realización de prácticas 
    intentando resolver las limitaciones que tienen otras plataformas para la gestión de prácticas.
  \end{center}
  \framebreak
  %+++++++++++++++++++++++++++++++++++++++++++++++++++++++++++++++++++++++++++++++++++++++++++++++++++++++++++++++++++++++++++++++++++++++++++
  
  \begin{itemize}
    \item El profesor podrá crear tareas que serán asignadas a sus alumnos de forma individual o grupal. 
    \item Cada alumno realizará su trabajo en un repositorio git, el cual una vez finalizado, podrá ser revisado por el profesor.
    \item La generación de aplicaciones correctoras de exámenes provistas de lo necesario para su despliegue y puesta en funcionamiento.
  \end{itemize}

  \framebreak
  %+++++++++++++++++++++++++++++++++++++++++++++++++++++++++++++++++++++++++++++++++++++++++++++++++++++++++++++++++++++++++++++++++++++++++++
  
  En los últimos años se han desarrollado las nuevas tecnologías, lo que ha permitido que lleguen 
  nuevas herramientas de aprendizaje y de apoyo a la docencia. Es el caso de la exitosa plataforma 
  Moodle, un LCMS, sigla de Learning Content Management System.
  
  \framebreak
  %+++++++++++++++++++++++++++++++++++++++++++++++++++++++++++++++++++++++++++++++++++++++++++++++++++++++++++++++++++++++++++++++++++++++++++
  
  La única herramienta para la gestión de tareas está desarrollada por Github, se trata de Github Classroom, 
  que forma parte de Github Education:
  
  \begin{itemize}
    \item GitHub Classroom
    \item Classroom Desktop
    \item Teachers Pet
    \item Student Pack
  \end{itemize}

  \framebreak
    %+++++++++++++++++++++++++++++++++++++++++++++++++++++++++++++++++++++++++++++++++++++++++++++++++++++++++++++++++++++++++++++++++++++++++++
    Classroom simplifica la asignación de tareas, automatizando la creación de repositorios, 
    es una herramienta útil y sencilla de usar, tanto para profesores como para alumnos, 
    pero tiene ciertos defectos:

    \begin{itemize}
      \item No se puede crear un repositorio de evaluación que contenga las tareas de todos los alumnos
      \item Tampoco se puede acceder a los enlaces de travis, si la práctica requiere su uso
      \item Tiene un sistema para asociar información a cada alumno que es bastante complejo de usar
    \end{itemize}

  \framebreak
    %+++++++++++++++++++++++++++++++++++++++++++++++++++++++++++++++++++++++++++++++++++++++++++++++++++++++++++++++++++++++++++++++++++++++++++
    
\end{frame}
  %+++++++++++++++++++++++++++++++++++++++++++++++++++++++++++++++++++++++++++++++++++++++++++++++++++++++++++++++++++++++++++++++++++++++++++

\section{Objetivos}
\begin{frame}[allowframebreaks,fragile]
  \frametitle{Objetivos}
  
  \begin{itemize}
    \item Analizar otras plataformas  web existentes para la gestión del código de las prácticas de informática y su metodología de trabajo.
    \item Estudiar el funcionamiento de otras plataformas  web existentes para la gestión del código de las prácticas de informática.
    \item Estudiar las tecnologías a usar y enfocar el diseño de la plataforma web.
    \item Estudiar las funcionalidades que se van a incluir en la plataforma web.
  \end{itemize}

  \framebreak

  \begin{itemize}
    \item Crear una aplicación web básica que permita al usuario iniciar sesión con su cuenta de Github.
    \item Continuar con el desarrollo de la aplicación incluyendo las funcionalidades que solucionen las dificultades de otras plataformas  web existentes para la gestión del código de las prácticas de informática.
    \item Diseñar y desarrollar los estilos de las vistas.
  \end{itemize}
  
\end{frame}

%++++++++++++++++++++++++++++++++++++++++++++++++++++++++++++++++++++++++++++++  

\section{Tecnologías usadas}
\begin{frame}[allowframebreaks,fragile]
  \frametitle{Tecnología usada}
  
  Se escogió Express.js, Node.js y Javascript como tecnologías principales a usar en este proyecto 
  por las siguientes razones:

  \begin{itemize}
    \item Express.js es bastante fácil de aprender y usar.
    \item Se usa JS tanto en Backend como en frontend, ahorrando tiempo de desarrollo.
    \item El gestor de paquetes NPM.
  \end{itemize}

  \framebreak

  Otras tecnologías usadas:

  \begin{itemize}
    \item Github API
    \item MongoDB - Mongoose
    \item El gestor de paquetes NPM.
    \item Pug.js
    \item Materialize
  \end{itemize}

\end{frame}

%++++++++++++++++++++++++++++++++++++++++++++++++++++++++++++++++++++++++++++++

\section{Desarrollo de la plataforma}
\begin{frame}
\frametitle{Desarrollo de la plataforma}
  
  \begin{center}
    \includegraphics[width=0.8\textwidth]{img/codelab.eps}
  \end{center}

\end{frame}

\subsection{Github API}
\begin{frame}
\frametitle{Github API}
  
  Para poder usar Github como estructura para la creación de aulas y tareas de código usaremos 
  la Github REST API V3. La API permite acceder a las funcionalidades de Github, 
  A continuación se detallará que funcionalidades de la Github API se han usado

  \begin{itemize}
    \item OAuth
    \item Organizaciones, repositorios y equipos.
  \end{itemize}
  
\end{frame}
  %+++++++++++++++++++++++++++++++++++++++++++++++++++++++++++++++++++++++++++++++++++++++++++++++++++++++++
  
\subsubsection{OAuth}
  
\begin{frame}
\frametitle{OAuth}

  Desde el punto de vista de Codelab, OAuth proporciona un método de acceso a los datos de Github 
  a la vez que se protegen los credenciales de la cuenta. Solicitamos al usuario una serie de 
  accesos y permisos en sus datos, para poder operar con las organizaciones y repositorios de Github.
  
\end{frame}
  %+++++++++++++++++++++++++++++++++++++++++++++++++++++++++++++++++++++++++++++++++++++++++++++++++++++++++++
  
\subsubsection{Organizaciones repositorios y equipos}

\begin{frame}
\frametitle{Organizaciones repositorios y equipos}

  Las operaciones que realizaremos con Organizaciones, repositorios y equipos son las siquientes:

  \begin{itemize}
    \item Obtener las organizaciones del usuario.
    \item Añadir usuarios a la organización.
    \item Crear un repositorio en las organizaciones.
    \item Añadir colaboradores al repositorio.
    \item Crear equipos.
    \item Añadir un equipo a un repositorio.
    \item Comprobar si un usuario es miembro de un equipo.
    \item Obtener los repositorios de una organización.
    \item Crear un fichero en un repositorio.
  \end{itemize}

\end{frame}

%++++++++++++++++++++++++++++++++++++++++++++++++++++++++++++++++++++++++++++++  
  
\subsection{MVC}

\begin{frame}
\frametitle{MVC}

  El modelo vista controlador es una arquitectura de software que separa la lógica de
  la aplicación de la interfaz de usuario. Lo hace separando la aplicación en tres partes: 
  el modelo, la vista y el controlador.

\end{frame}

%+++++++++++++++++++++++++++++++++++++++++++++++++++++++++++++++++++++++++++++++++++++++++++++++++++++++++++

\subsection{Diseño de la base de datos}

\begin{frame}[allowframebreaks]
\frametitle{Diseño de la base de datos}

  Se usará una base de datos NoSQL, en concreto MongoDB como sistema gestor y Mongoose como ODM, 
  por lo que nos referimos las tablas como colecciones, columnas como claves y filas como objeto.
  
  \bigskip
  
  La base de datos es una parte muy importante de la plataforma, ya que en ella recae toda 
  la responsabilidad de simular toda la estructura de aulas y tareas.

  \framebreak

    Las colecciones son las siguientes:

    \begin{itemize}
      \item Usuarios
      \item Organizaciones
      \item Asignaciones
      \item Asignaciones en grupo
      \item Asignaciones individuales
      \item Equipos
      \item Alumnos
    \end{itemize}

\end{frame}

%+++++++++++++++++++++++++++++++++++++++++++++++++++++++++++++++++++++++++++++++++++++++++++++++++++++++++++

\section{Funcionalidades}

\begin{frame}[allowframebreaks]
\frametitle{Funcionalidades}

  El proyecto se divide en tres paquetes de funcionalidades, Cada paquete incluye funcionalidades para cada rol:

  \begin{itemize}
    \item Funcionalidades básicas
    \item Funcionalidades para profesores.
    \item Funcionalidades para el alumno
  \end{itemize}

  \framebreak
  
  Las funcionalidades básicas son comunes a todos los roles que participan en la plataforma, 
  tanto alumnos como profesores, todos pueden hacer Log in, Log out y consultar un perfil.
  
  \bigskip
  
  Como alumno el usuario puede visitar el perfil, donde encontrará información básica de Github y dos pestañas, 
  en las que el alumno tiene un historial de las tareas que ha realizado de forma grupal e individual.
  
  \bigskip

  Los profesores tendrán el grupo de funcionalidades más completo, ya que ellos son los protagonistas 
  de la app.

  \framebreak

  Los profesores podrán desempeñar las siguientes tareas:

  \begin{itemize}
    \item Añadir una organización como aula.
    \item Invitar alumnos al aula.
    \item Crear una tarea.
    \item Añadir un fichero de alumnos asociado al aula.
    \item Editar las opciones del aula.
    \item Invitar alumnos a la tarea.
    \item Editar las opciones de la tarea.
    \item Crear un repositorio de evaluación de cada tarea.
  \end{itemize}

\end{frame}

%+++++++++++++++++++++++++++++++++++++++++++++++++++++++++++++++++++++++++++++++++++++++++++++++++++++++++++
  
\subsection{Caso de uso}

\begin{frame}[allowframebreaks]
\frametitle{Caso de uso}

  Con vistas a probar y testear que todo funcionaba de forma correcta, Casiano me sugirió probar 
  la plataforma para la realización de algunas prácticas individuales y grupales. Por ello, decidimos 
  realizar algunas tareas para la asignatura de Procesadores de Lenguajes en CodeLab.

  \framebreak

  \begin{figure}[!htb]
    \minipage{0.32\textwidth}
      \includegraphics[width=\linewidth]{img/gh.eps}
    \endminipage\hfill
    \minipage{0.32\textwidth}
      \includegraphics[width=\linewidth]{img/teams.eps}
    \endminipage\hfill
    \minipage{0.32\textwidth}%
      \includegraphics[width=\linewidth]{img/campus.eps}
    \endminipage
  \end{figure}

\end{frame}

%+++++++++++++++++++++++++++++++++++++++++++++++++++++++++++++++++++++++++++++++++++++++++++++++++++++++++++

\section{Conclusiones y Trabajos Futuros/Conclusions and Future Work}
\begin{frame}[allowframebreaks]
  \frametitle{Conclusiones y Trabajos Futuros/Conclusions and Future Work}
  
  \begin{itemize}
    \item Esta herramienta pretende ser un complemento para plataformas como Moodle al ofrecer la posibilidad de
    especificar preguntas de programación.
    \item Ofrece una solución innovadora para el almacenamiento de los datos de los exámenes: {\bfseries Google Drive}.
    Permite gestionar de forma más cómoda los datos generados en lugar de usar bases de datos.
    \item Por otra parte, considerando aspectos éticos y de seguridad, se hace uso de {\bfseries OAuth} para la autentificación de usuarios con el fin de evitar
    posibles problemas de phishing y exposición de datos sensibles a terceras personas.
  \end{itemize}
  \framebreak
  %+++++++++++++++++++++++++++++++++++++++++++++++++++++++++++++++++++++++++++++++++++++++++++++++++++++++++++++++++++++++++++++++++++++++++++
  
  {\bf Trabajos Futuros:}
  \begin{itemize}
    \item Resolver los problemas de seguridad relacionados con evaluar el código escrito de los alumnos.
    \item Dar soporte a preguntas con respuestas de código en otros lenguajes de programación.
    \item Ofrecer una alternativa de despliegue distinta a Heroku.
    \item Escribir \textit{renderers} para dar soporte a otros formatos usados por diversas plataformas educativas (Ej: MoodleXML, Gift, etc.).
  \end{itemize}
  \framebreak
  %+++++++++++++++++++++++++++++++++++++++++++++++++++++++++++++++++++++++++++++++++++++++++++++++++++++++++++++++++++++++++++++++++++++++++++
  
  \begin{itemize}
    \item This tool intends to complement learning management systems with new capabilities like the possibility to specify {\bfseries programming questions}.
    \item It uses {\bfseries Google Drive} for the storage of exams data (instead using databases),
    providing in this way a more natural solution from the lecturer perspective.
    \item On the other hand, keeping in mind ethic and legal topics, we use {\bfseries OAuth} to delegate the authentication to Google. 
    This way, we avoid security bugs as the phishing or the exposure of sensitive information to third people.
  \end{itemize}
  \framebreak
  %+++++++++++++++++++++++++++++++++++++++++++++++++++++++++++++++++++++++++++++++++++++++++++++++++++++++++++++++++++++++++++++++++++++++++++
  
  {\bf Future Work:}
  \begin{itemize}
    \item Solve the security problem related with the evaluation of student code in the server.
    \item Provide support to questions with answers written in other programming languages.
    \item Provide a deployment alternative different to Heroku.
    \item To write renderers giving support to other formats (MoodleXML, Gift, etc.) used by a variety of learning platforms.
  \end{itemize}
\end{frame}

%++++++++++++++++++++++++++++++++++++++++++++++++++++++++++++++++++++++++++++++ 

\section{Bibliografía}
\begin{frame}[allowframebreaks]
  \frametitle{Bibliografía}
  \bibliographystyle{ieeetr}
  \bibliography{presentacion_tfg}
  \nocite{*}
\end{frame}

\begin{frame}
  \frametitle{Fin de la presentación}
  \begin{center}
    \Huge{Gracias por su atención}
  \end{center}
\end{frame}

\end{document}
