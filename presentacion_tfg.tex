\documentclass{beamer}
%\documentclass[xcolor=dvipsnames]{beamer}
\usepackage[spanish]{babel}
\usepackage[utf8]{inputenc}
\usepackage{graphicx}
\usepackage{latexsym}

\newcommand{\beamer}{\textsc{beamer}}
\newtheorem{definicion}{Definición}
\newtheorem{ejemplo}{Ejemplo}

%%%%%%%%%%%%%%%%%%%%%%%%%%%%%%%%%%%%%%%%%%%%%%%%%%%%%%%%%%%%%%%%%%%%%%%%%%%%%%%
\title[Trabajo de Fin de Grado]{CodeLab\\
A Tool to automate repository creation and access control, giving support to distribute starter code, collect assignments and evaluate the students work on GitHub.}

\author[Samuel Ramos Barroso] {
Autor: Samuel Ramos Barroso \\
Director: Casiano Rodríguez León
}

\institute[ULL]{Escuela Superior de Ingeniería y Tecnología \\
                Departamento de Ingeniería Informática y de Sistemas \\
                Universidad de La Laguna}
\date[14-06-2018]{14 de Junio de 2018}
%%%%%%%%%%%%%%%%%%%%%%%%%%%%%%%%%%%%%%%%%%%%%%%%%%%%%%%%%%%%%%%%%%%%%%%%%%%%%%%

%\usetheme{Berlin}
\usetheme{Madrid}

%%%%%%%%%%%%%%%%%%%%%%%%%%%%%%%%%%%%%%%%%%%%%%%%%%%%%%%%%%%%%%%%%%%%%%%%%%%%%%%
\definecolor{pantone254}{RGB}{92,6,140}
\definecolor{pantone3015}{RGB}{92,6,140}
\definecolor{pantone432}{RGB}{92,6,140}
\setbeamercolor*{palette primary}{use=structure,fg=white,bg=pantone254}
\setbeamercolor*{palette secondary}{use=structure,fg=white,bg=pantone3015}
\setbeamercolor*{palette tertiary}{use=structure,fg=white,bg=pantone432}
\setbeamercolor*{palette sidebar primary}{use=structure,fg=pantone254}
\setbeamercolor*{palette sidebar tertiary}{use=structure,fg=pantone3015}
\setbeamercolor*{block title}{bg=pantone3015,fg=white}
\setbeamercolor*{alerted text}{fg=pantone432}
\setbeamercolor*{item projected}{fg=pantone254}
\setbeamercolor*{section in toc shaded}{use=structure,fg=structure.fg}
\setbeamercolor*{section in toc}{fg=pantone3015}
\setbeamercolor*{subsection in toc shaded}{fg=pantone3015}
\setbeamercolor*{subsection in toc}{fg=pantone432}

%%%%%%%%%%%%%%%%%%%%%%%%%%%%%%%%%%%%%%%%%%%%%%%%%%%%%%%%%%%%%%%%%%%%%%%%%%%%%%%
\begin{document}
  
%++++++++++++++++++++++++++++++++++++++++++++++++++++++++++++++++++++++++++++++  
\begin{frame}

  \includegraphics[width=0.3\textwidth]{img/ull.eps}
  \hspace*{7.5cm}
  \titlepage

\end{frame}
%++++++++++++++++++++++++++++++++++++++++++++++++++++++++++++++++++++++++++++++  

%++++++++++++++++++++++++++++++++++++++++++++++++++++++++++++++++++++++++++++++  
\begin{frame}
  \frametitle{Índice}  
  \tableofcontents
\end{frame}
%++++++++++++++++++++++++++++++++++++++++++++++++++++++++++++++++++++++++++++++  

\section{Introducción}
\begin{frame}[allowframebreaks,fragile]
  \frametitle{Introducción}
  
  \begin{center}
    CodeLab es una plataforma web basada en Javascript destinada al apoyo del profesorado para la realización de prácticas 
    intentando resolver las limitaciones que tienen otras plataformas para la gestión de prácticas.
  \end{center}
  \framebreak
  %+++++++++++++++++++++++++++++++++++++++++++++++++++++++++++++++++++++++++++++++++++++++++++++++++++++++++++++++++++++++++++++++++++++++++++
  
  \begin{itemize}
    \item El profesor podrá crear tareas que serán asignadas a sus alumnos de forma individual o grupal. 
    \item Cada alumno realizará su trabajo en un repositorio git, el cual una vez finalizado, podrá ser revisado por el profesor.
    \item La generación de aplicaciones correctoras de exámenes provistas de lo necesario para su despliegue y puesta en funcionamiento.
  \end{itemize}

  \framebreak
  %+++++++++++++++++++++++++++++++++++++++++++++++++++++++++++++++++++++++++++++++++++++++++++++++++++++++++++++++++++++++++++++++++++++++++++
  
  En los últimos años se han desarrollado las nuevas tecnologías, lo que ha permitido que lleguen 
  nuevas herramientas de aprendizaje y de apoyo a la docencia. Es el caso de la exitosa plataforma 
  Moodle, un LCMS, sigla de Learning Content Management System.
  
  \framebreak
  %+++++++++++++++++++++++++++++++++++++++++++++++++++++++++++++++++++++++++++++++++++++++++++++++++++++++++++++++++++++++++++++++++++++++++++
  
  La única herramienta para la gestión de tareas está desarrollada por Github, se trata de Github Classroom, 
  que forma parte de Github Education:
  
  \begin{itemize}
    \item GitHub Classroom
    \item Classroom Desktop
    \item Teachers Pet
    \item Student Pack
  \end{itemize}

  \framebreak
    %+++++++++++++++++++++++++++++++++++++++++++++++++++++++++++++++++++++++++++++++++++++++++++++++++++++++++++++++++++++++++++++++++++++++++++
    Classroom simplifica la asignación de tareas, automatizando la creación de repositorios, 
    es una herramienta útil y sencilla de usar, tanto para profesores como para alumnos, 
    pero tiene ciertos defectos:

    \begin{itemize}
      \item No se puede crear un repositorio de evaluación que contenga las tareas de todos los alumnos
      \item Tampoco se puede acceder a los enlaces de travis, si la práctica requiere su uso
      \item Tiene un sistema para asociar información a cada alumno que es bastante complejo de usar
    \end{itemize}

  \framebreak
    %+++++++++++++++++++++++++++++++++++++++++++++++++++++++++++++++++++++++++++++++++++++++++++++++++++++++++++++++++++++++++++++++++++++++++++
    
\end{frame}
  %+++++++++++++++++++++++++++++++++++++++++++++++++++++++++++++++++++++++++++++++++++++++++++++++++++++++++++++++++++++++++++++++++++++++++++

\section{Objetivos}
\begin{frame}[allowframebreaks,fragile]
  \frametitle{Objetivos}
  
  \begin{itemize}
    \item Analizar otras plataformas  web existentes para la gestión del código de las prácticas de informática y su metodología de trabajo.
    \item Estudiar el funcionamiento de otras plataformas  web existentes para la gestión del código de las prácticas de informática.
    \item Estudiar las tecnologías a usar y enfocar el diseño de la plataforma web.
    \item Estudiar las funcionalidades que se van a incluir en la plataforma web.
  \end{itemize}

  \framebreak

  \begin{itemize}
    \item Crear una aplicación web básica que permita al usuario iniciar sesión con su cuenta de Github.
    \item Continuar con el desarrollo de la aplicación incluyendo las funcionalidades que solucionen las dificultades de otras plataformas  web existentes para la gestión del código de las prácticas de informática.
    \item Diseñar y desarrollar los estilos de las vistas.
  \end{itemize}
  
\end{frame}

%++++++++++++++++++++++++++++++++++++++++++++++++++++++++++++++++++++++++++++++  

\section{Tecnologías usadas}
\begin{frame}
  \frametitle{Tecnología usada}
  
  Se escogió Express.js, Node.js y Javascript como tecnologías principales a usar en este proyecto 
  por las siguientes razones:

  \begin{itemize}
    \item Express.js es bastante fácil de aprender y usar.
    \item Se usa JS tanto en Backend como en frontend, ahorrando tiempo de desarrollo.
    \item El gestor de paquetes NPM.
  \end{itemize}

\end{frame}

%++++++++++++++++++++++++++++++++++++++++++++++++++++++++++++++++++++++++++++++  

\section{Metodología de desarrollo}
\begin{frame}
  \frametitle{Metodología de desarrollo}
  
  Metodología {\bfseries ágil}:
  \begin{itemize}
    \item Reuniones semanales estableciendo iteraciones cortas.
    \item Desarrollo, testing y presentación de resultados y prototipos cada semana.
    \item Solución de problemas e incorporación de nuevas características. 
  \end{itemize}
  \bigskip
  
  {\bfseries GitHub}:
  \begin{columns}
    % First column
    \begin{column}{9cm}
      \begin{itemize}
        \item Control de versiones usando \textit{branching}.
        \item Gestión de incidencias y mejoras usando \textit{issues}.
        \item Contacto con Armando para los \textit{Pull Requests}.
      \end{itemize}
    \end{column}
    % Second column
    \begin{column}{4cm}
      \begin{figure}
        \includegraphics[width=0.7\textwidth]{img/octocat.eps}
      \end{figure}
    \end{column}
  \end{columns}
\end{frame}

%++++++++++++++++++++++++++++++++++++++++++++++++++++++++++++++++++++++++++++++

\section{Resultados}
\begin{frame}
\frametitle{Resultados}
  
  \begin{center}
    \Huge{Resultados}
  \end{center}
\end{frame}

\subsection{Corrección de errores y mejoras de la gema original}
\begin{frame}
\frametitle{Corrección de errores y mejoras de la gema original}
  \bigskip
  \bigskip
  
  \begin{itemize}
    \item Corrección de errores de funcionamiento de la gema.
    \item Corrección de tests.
    \item Refactorización de código.
    \item Añadido manejo de excepciones y mejora de los mensajes de error.
  \end{itemize}
  
  \begin{figure}
    \hfill\includegraphics[width=0.2\textwidth]{img/tick.eps}
  \end{figure}
  
\end{frame}
  %+++++++++++++++++++++++++++++++++++++++++++++++++++++++++++++++++++++++++++++++++++++++++++++++++++++++++
  
\subsection{Cuestionarios de entrenamiento para alumnos: HtmlForm renderer}
  
\begin{frame}[allowframebreaks]
\frametitle{Cuestionarios de entrenamiento para alumnos: HtmlForm renderer}
  Se genera un fichero HTML que contiene un formulario web. 
  \bigskip
  
  Características destacadas:
  \begin{itemize}
    \item Validación por {\bfseries JavaScript}.
    \item {\bfseries Local Storage} de HTML5 para almacenar las respuestas introducidas.
    \item Expresiones regulares más potentes usando \includegraphics[width=0.23\textwidth]{img/xregexp.eps}
    \framebreak
    %+++++++++++++++++++++++++++++++++++++++++++++++++++++++++++++++++++++++++++++++++++++++++
    
    \item Soporte a preguntas de completar que evalúan código JavaScript.
    \bigskip
    \begin{center}
      \includegraphics[width=0.8\textwidth]{img/fi_p.eps}
    \end{center}
    \bigskip
    
    \item Soporte a expresiones escritas en {\bfseries LaTeX} usando {\bfseries MathJax}.
    \bigskip
    \begin{center}
      \includegraphics[width=0.8\textwidth]{img/latex.eps}
    \end{center}
    \framebreak
    %+++++++++++++++++++++++++++++++++++++++++++++++++++++++++++++++++++++++++++++++++++++++++
    
    \item Soporte a preguntas de programación (código JavaScript).
    \bigskip
    \begin{center}
      \includegraphics[width=0.9\textwidth]{img/programming.eps}
    \end{center}
    \framebreak
    %+++++++++++++++++++++++++++++++++++++++++++++++++++++++++++++++++++++++++++++++++++++++++
    
    \item Soporte a preguntas de Drag and Drop.
    \bigskip
    \begin{columns}
      % First column
      \begin{column}{7cm}
        \includegraphics[width=0.9\textwidth]{img/ddfi.eps}
        \newline
        \newline
        \includegraphics[width=0.9\textwidth]{img/ddmc.eps}
      \end{column}
      % Second column
      \begin{column}{5.5cm}
        \includegraphics[width=1\textwidth]{img/ddsm.eps}
      \end{column}
    \end{columns}
    \framebreak
    %+++++++++++++++++++++++++++++++++++++++++++++++++++++++++++++++++++++++++++++++++++++++++
    
    \item Posibilidad de mostrar las respuestas correctas
    \bigskip
    \begin{itemize}
      \item Mediante menú contextual:
      \begin{center}
        \includegraphics[width=0.5\textwidth]{img/show_answer.eps}
      \end{center}
      \item Mediante botones:
      \begin{center}
        \includegraphics[width=0.5\textwidth]{img/sm.eps}
      \end{center}  
    \end{itemize}
    
  \end{itemize}
  \framebreak
  %+++++++++++++++++++++++++++++++++++++++++++++++++++++++++++++++++++++++++++++++++++++++++
  
  {\bfseries {\Large Funcionamiento}}
  \bigskip  
  \bigskip
  
  \begin{columns}
    % First column
    \begin{column}{3cm}
      \includegraphics[width=0.7\textwidth]{img/ruby_file.eps} \\
      \hspace*{0.1cm}
      example.rb
    \end{column}
    % Second column
    \begin{column}{0.1px}
      $\rightarrow$
    \end{column}
    % Third column
    \begin{column}{5cm}
      \hspace*{0.8cm}
      \begin{center}
        \includegraphics[width=0.55\textwidth]{img/terminal_icon.eps} \\
        \tiny{\textit{[\textasciitilde]\$ ruql example.rb HtmlForm \textgreater \,output.html}}
      \end{center}
    \end{column}
    % Fourth column
    \begin{column}{0.5cm}
      $\rightarrow$
    \end{column}
    % Fifth column
    \begin{column}{3cm}
      \includegraphics[width=0.7\textwidth]{img/file_html.eps} \\
      output.html
    \end{column}
  \end{columns}
  
\end{frame}
  %+++++++++++++++++++++++++++++++++++++++++++++++++++++++++++++++++++++++++++++++++++++++++++++++++++++++++++
  
\subsection{Aplicación correctora de exámenes: Sinatra renderer}

\begin{frame}[allowframebreaks]
\frametitle{Aplicación correctora de exámenes: Sinatra renderer}
  Genera una aplicación Sinatra con todo lo necesario para ser desplegada.
  \bigskip
  
  Características:
  \begin{itemize}
    \item Roles de usuarios: profesores y alumnos.
    \item Autenticación usando {\bfseries OAuth} con las cuentas de Gmail.
    \item Ventana temporal en la que el cuestionario estará disponible.
    \item Corrección de los cuestionarios realizados por los alumnos.
    \item Soporte a los tipos de preguntas explicados anteriormente.
    \item Soporte a preguntas de programación (código en {\bfseries Ruby}).
    \framebreak
    %+++++++++++++++++++++++++++++++++++++++++++++++++++++++++++++++++++++++++++++++++++++++++++++++++++++++++++
    
    \begin{center}
      {\bfseries {\Large Funcionamiento}}
    \end{center}
    
    \begin{columns}
      % First column
      \begin{column}{3.05cm}
        \hspace*{0.3cm}
        \includegraphics[width=0.5\textwidth]{img/ruby_file.eps} \\
        \hspace*{0.3cm}
        example.rb
        \newline
        \newline
        \hspace*{0.3cm}
        \includegraphics[width=0.5\textwidth]{img/yml_file.eps} \\
        \hspace*{0.3cm}
        config.yml
        \newline
        \newline
        \hspace*{0.3cm}
        \includegraphics[width=0.5\textwidth]{img/csv_file.eps} \\
        \hspace*{0.3cm}
        students.csv
      \end{column}
      % Second column
      \begin{column}{0.1px}
        $\rightarrow$
      \end{column}
      % Third column
      \begin{column}{4cm}
        \hspace*{0.8cm}
        \begin{center}
          \includegraphics[width=0.6\textwidth]{img/terminal_icon.eps} \\
          \tiny{\textit{[\textasciitilde]\$ ruql example.rb Sinatra}}
        \end{center}
      \end{column}
      % Fourth column
      \begin{column}{0.5cm}
        $\rightarrow$
      \end{column}
      % Fifth column
      \begin{column}{4cm}
        \begin{center}
        \end{center}
      \end{column}
    \end{columns}
    
    \framebreak
    %+++++++++++++++++++++++++++++++++++++++++++++++++++++++++++++++++++++++++++++++++++++++++++++++++++++++++++
    
    \begin{columns}
      % First column
      \begin{column}{7cm}
        \includegraphics[width=1\textwidth]{img/yml.eps} \\
        \hspace*{1cm}
        config.yml
      \end{column}
      % Second column
      \begin{column}{5cm}
        \includegraphics[width=1\textwidth]{img/quiz_sinatra.eps} \\
        \hspace*{1cm}
        example.rb
      \end{column}
    \end{columns}
    \bigskip
    
    \begin{center}
      \includegraphics[width=0.7\textwidth]{img/csv.eps} \\
      students.csv
    \end{center}
    
    \framebreak
    %+++++++++++++++++++++++++++++++++++++++++++++++++++++++++++++++++++++++++++++++++++++++++++++++++++++++++++
    
    \item Almacenamiento del cuestionario, respuestas y notas de los alumnos en {\bfseries Google Drive}.
    \bigskip
    \begin{center}
      \includegraphics[width=0.9\textwidth]{img/app0.eps}
      \newline
      \newline
      \includegraphics[width=0.5\textwidth]{img/app6.eps}
    \end{center}
    
    \framebreak
    %+++++++++++++++++++++++++++++++++++++++++++++++++++++++++++++++++++++++++++++++++++++++++++++++++++++++++++
    
    \begin{center}
      \includegraphics[width=0.8\textwidth]{img/app3.eps}
    \end{center}
    \framebreak
    %+++++++++++++++++++++++++++++++++++++++++++++++++++++++++++++++++++++++++++++++++++++++++++++++++++++++++++
    
    \begin{center}
      \includegraphics[width=0.6\textwidth]{img/app4.eps}
    \end{center}
    \framebreak
    %+++++++++++++++++++++++++++++++++++++++++++++++++++++++++++++++++++++++++++++++++++++++++++++++++++++++++++
    
    \begin{center}
      \includegraphics[width=0.9\textwidth]{img/app1.eps} \\
      \bigskip
      \begin{columns}
        \begin{column}{4cm}
          \includegraphics[width=1.2\textwidth]{img/app2.eps}
        \end{column}
        \begin{column}{4cm}
          \includegraphics[width=1.1\textwidth]{img/app7.eps}
        \end{column}
      \end{columns}
    \end{center}
    \framebreak
    %+++++++++++++++++++++++++++++++++++++++++++++++++++++++++++++++++++++++++++++++++++++++++++++++++++++++++++
    
    \begin{center}
      \includegraphics[width=0.5\textwidth]{img/app5.eps}
    \end{center}
    
  \end{itemize}
\end{frame}
%++++++++++++++++++++++++++++++++++++++++++++++++++++++++++++++++++++++++++++++  

\section{Conclusiones y Trabajos Futuros/Conclusions and Future Work}
\begin{frame}[allowframebreaks]
  \frametitle{Conclusiones y Trabajos Futuros/Conclusions and Future Work}
  
  \begin{itemize}
    \item Esta herramienta pretende ser un complemento para plataformas como Moodle al ofrecer la posibilidad de
    especificar preguntas de programación.
    \item Ofrece una solución innovadora para el almacenamiento de los datos de los exámenes: {\bfseries Google Drive}.
    Permite gestionar de forma más cómoda los datos generados en lugar de usar bases de datos.
    \item Por otra parte, considerando aspectos éticos y de seguridad, se hace uso de {\bfseries OAuth} para la autentificación de usuarios con el fin de evitar
    posibles problemas de phishing y exposición de datos sensibles a terceras personas.
  \end{itemize}
  \framebreak
  %+++++++++++++++++++++++++++++++++++++++++++++++++++++++++++++++++++++++++++++++++++++++++++++++++++++++++++++++++++++++++++++++++++++++++++
  
  {\bf Trabajos Futuros:}
  \begin{itemize}
    \item Resolver los problemas de seguridad relacionados con evaluar el código escrito de los alumnos.
    \item Dar soporte a preguntas con respuestas de código en otros lenguajes de programación.
    \item Ofrecer una alternativa de despliegue distinta a Heroku.
    \item Escribir \textit{renderers} para dar soporte a otros formatos usados por diversas plataformas educativas (Ej: MoodleXML, Gift, etc.).
  \end{itemize}
  \framebreak
  %+++++++++++++++++++++++++++++++++++++++++++++++++++++++++++++++++++++++++++++++++++++++++++++++++++++++++++++++++++++++++++++++++++++++++++
  
  \begin{itemize}
    \item This tool intends to complement learning management systems with new capabilities like the possibility to specify {\bfseries programming questions}.
    \item It uses {\bfseries Google Drive} for the storage of exams data (instead using databases),
    providing in this way a more natural solution from the lecturer perspective.
    \item On the other hand, keeping in mind ethic and legal topics, we use {\bfseries OAuth} to delegate the authentication to Google. 
    This way, we avoid security bugs as the phishing or the exposure of sensitive information to third people.
  \end{itemize}
  \framebreak
  %+++++++++++++++++++++++++++++++++++++++++++++++++++++++++++++++++++++++++++++++++++++++++++++++++++++++++++++++++++++++++++++++++++++++++++
  
  {\bf Future Work:}
  \begin{itemize}
    \item Solve the security problem related with the evaluation of student code in the server.
    \item Provide support to questions with answers written in other programming languages.
    \item Provide a deployment alternative different to Heroku.
    \item To write renderers giving support to other formats (MoodleXML, Gift, etc.) used by a variety of learning platforms.
  \end{itemize}
\end{frame}

%++++++++++++++++++++++++++++++++++++++++++++++++++++++++++++++++++++++++++++++ 

\section{Bibliografía}
\begin{frame}[allowframebreaks]
  \frametitle{Bibliografía}
  \bibliographystyle{ieeetr}
  \bibliography{presentacion_tfg}
  \nocite{*}
\end{frame}

\begin{frame}
  \frametitle{Fin de la presentación}
  \begin{center}
    \Huge{Gracias por su atención}
  \end{center}
\end{frame}

\end{document}
